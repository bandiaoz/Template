\documentclass[a4paper,11pt]{ctexbook} 
\usepackage[a4paper,scale=0.8,bindingoffset=8mm]{geometry} % A4纸大小,缩放80%,设置奇数页右边留空多一点
\usepackage{hyperref}      % 超链接
\usepackage{listings}      % 代码块
\usepackage{fancyhdr}      % 页眉页脚相关宏包
\usepackage{lastpage}      % 引用最后一页
\usepackage{amsmath,amsthm,amsfonts,amssymb,bm} %数学
\usepackage{graphicx}      % 图片
\usepackage{subcaption}    % 图片描述
\usepackage{longtable,booktabs} % 表格
\usepackage{xeCJK}

\usepackage{listings}
\usepackage{ctex}
\usepackage{color,xcolor}

% 用来设置附录中代码的样式



\lstset{
    basicstyle          =   \sffamily,          % 基本代码风格
    keywordstyle        =   \bfseries,          % 关键字风格
    commentstyle        =   \rmfamily\itshape,  % 注释的风格,斜体
    stringstyle         =   \ttfamily,  % 字符串风格
    flexiblecolumns,                % 别问为什么,加上这个
    numbers             =   left,   % 行号的位置在左边
    showspaces          =   false,  % 是否显示空格,显示了有点乱,所以不现实了
    numberstyle         =   \zihao{-5}\ttfamily,    % 行号的样式,小五号,tt等宽字体
    showstringspaces    =   false,
    captionpos          =   t,      % 这段代码的名字所呈现的位置,t指的是top上面
    frame               =   l,   % 显示边框
}

\lstdefinestyle{Python}{
    language        =   Python, % 语言选Python
    basicstyle      =   \zihao{-5}\ttfamily,
    numberstyle     =   \zihao{-5}\ttfamily,
    keywordstyle    =   \color{blue},
    keywordstyle    =   [2] \color{teal},
    stringstyle     =   \color{magenta},
    commentstyle    =   \color{red}\ttfamily,
    breaklines      =   true,   % 自动换行,建议不要写太长的行
    columns         =   fixed,  % 如果不加这一句,字间距就不固定,很丑,必须加
    basewidth       =   0.5em,
}
\lstdefinestyle{cpp}{
    language        =   C++, % 语言选C++
    basicstyle      =   \zihao{-5}\ttfamily,
    numberstyle     =   \zihao{-5}\ttfamily,
    % keywordstyle    =   \color{blue}, % 彩色
    % keywordstyle    =   [2] \color{teal},% 彩色
    keywordstyle    =   \color{black}, % 黑白
    keywordstyle    =   [2] \color{black},% 黑白
    % stringstyle     =   \color{magenta}, % 彩色
    stringstyle     =   \color{black}, % 黑白
    % commentstyle    =   \color{red}\ttfamily,% 彩色
    commentstyle    =   \color{black}\ttfamily, % 黑白
    breaklines      =   true,   % 自动换行,建议不要写太长的行
    columns         =   fixed,  % 如果不加这一句,字间距就不固定,很丑,必须加
    basewidth       =   0.5em,
}

%\lstinputlisting[
%     style       =   Python,
%     caption     =   {\bf ff.py},
%     label       =   {ff.py}
% ]{../src/duke/ff.py}


\pagestyle{fancy}         % 页眉页脚风格
\fancyhf{}                % 清空当前设置
\fancyfoot[C]{\thepage\ / \pageref{LastPage}}%页脚中间显示 当前页 / 总页数,把\label{LastPage}放在最后
\fancyhead[RO,LE]{\thepage}% 页眉奇数页左边,偶数页右边显示当前页
\begin{document} 
  \begin{titlepage}       % 封面
    \centering
    \vspace*{\baselineskip}
    \rule{\textwidth}{1.6pt}\vspace*{-\baselineskip}\vspace*{2pt}
    \rule{\textwidth}{0.4pt}\\[\baselineskip]
    {\LARGE Algos @bandiaoz 2024\\[\baselineskip]\small for ACM}
    \\[0.2\baselineskip]
    \rule{\textwidth}{0.4pt}\vspace*{-\baselineskip}\vspace{3.2pt}
    \rule{\textwidth}{1.6pt}\\[\baselineskip]
    \scshape

    \begin{figure}[!htb]
        \centering
        %\includegraphics[width=0.3\textwidth]{icpc}    % 当前tex文件同一目录下名为icpc的任意格式图片
    \end{figure}

    % \vspace*{3\baselineskip}
    % Edited by \\[\baselineskip] {Byjiang\par}
    % {Team \Large Byjiang \normalsize{at CUGB}\par }
    \vfill
    {\scshape 2024} \\{\large bandiaoz}\par
  \end{titlepage}
 \newpage            % 封面背后空白页
\tableofcontents     % 目录

\setcounter{page}{1}

\section{DataStruct}
\subsection{TheKthFarPointPair.cpp}
\lstinputlisting[style=cpp]{../../src/DataStruct/TheKthFarPointPair.cpp}
\subsection{NearestPointPair.cpp}
\lstinputlisting[style=cpp]{../../src/DataStruct/NearestPointPair.cpp}
\subsection{WaveLet.h}
\lstinputlisting[style=cpp]{../../src/DataStruct/WaveLet.h}
\subsection{AdjDiff2D.h}
\lstinputlisting[style=cpp]{../../src/DataStruct/AdjDiff2D.h}
\subsection{dsu\_on\_tree.cpp}
\lstinputlisting[style=cpp]{../../src/DataStruct/dsu_on_tree.cpp}
\subsection{MinMax.hpp}
\lstinputlisting[style=cpp]{../../src/DataStruct/MinMax.hpp}
\subsection{BiTrie.h}
\lstinputlisting[style=cpp]{../../src/DataStruct/BiTrie.h}
\subsection{Mo.cpp}
\lstinputlisting[style=cpp]{../../src/DataStruct/misc/Mo.cpp}
\subsection{RangeManger.h}
\lstinputlisting[style=cpp]{../../src/DataStruct/misc/RangeManger.h}
\subsection{Hash.cpp}
\lstinputlisting[style=cpp]{../../src/DataStruct/misc/Hash.cpp}
\subsection{Chtholly.cpp}
\lstinputlisting[style=cpp]{../../src/DataStruct/misc/Chtholly.cpp}
\subsection{SparseTable.h}
\lstinputlisting[style=cpp]{../../src/DataStruct/RMQ/SparseTable.h}
\subsection{MaskRMQ.h}
\lstinputlisting[style=cpp]{../../src/DataStruct/RMQ/MaskRMQ.h}
\subsection{LinearDSU.h}
\lstinputlisting[style=cpp]{../../src/DataStruct/DSU/LinearDSU.h}
\subsection{PotentializedDSU.h}
\lstinputlisting[style=cpp]{../../src/DataStruct/DSU/PotentializedDSU.h}
\subsection{DSU.hpp}
\lstinputlisting[style=cpp]{../../src/DataStruct/DSU/DSU.hpp}
\subsection{StaticBufferWrapWithoutCollect.h}
\lstinputlisting[style=cpp]{../../src/DataStruct/container/StaticBufferWrapWithoutCollect.h}
\subsection{ErasableMinMaxHeap.h}
\lstinputlisting[style=cpp]{../../src/DataStruct/container/ErasableMinMaxHeap.h}
\subsection{MinMaxHeap.h}
\lstinputlisting[style=cpp]{../../src/DataStruct/container/MinMaxHeap.h}
\subsection{StaticBufferWrapWithCollect.h}
\lstinputlisting[style=cpp]{../../src/DataStruct/container/StaticBufferWrapWithCollect.h}
\subsection{VectorBufferWithCollect.h}
\lstinputlisting[style=cpp]{../../src/DataStruct/container/VectorBufferWithCollect.h}
\subsection{VectorBufferWithoutCollect.h}
\lstinputlisting[style=cpp]{../../src/DataStruct/container/VectorBufferWithoutCollect.h}
\subsection{FastHeap.h}
\lstinputlisting[style=cpp]{../../src/DataStruct/container/FastHeap.h}
\subsection{SiftHeap.h}
\lstinputlisting[style=cpp]{../../src/DataStruct/container/SiftHeap.h}
\subsection{CountColor.cpp}
\lstinputlisting[style=cpp]{../../src/DataStruct/ClassicProblem/CountColor.cpp}
\subsection{RangeMex.h}
\lstinputlisting[style=cpp]{../../src/DataStruct/ClassicProblem/RangeMex.h}
\subsection{OfflinePointAddRectSumCounter2D.h}
\lstinputlisting[style=cpp]{../../src/DataStruct/ClassicProblem/OfflinePointAddRectSumCounter2D.h}
\subsection{PointAddRectSumCounter2D.h}
\lstinputlisting[style=cpp]{../../src/DataStruct/ClassicProblem/PointAddRectSumCounter2D.h}
\subsection{RangeMode.h}
\lstinputlisting[style=cpp]{../../src/DataStruct/ClassicProblem/RangeMode.h}
\subsection{OfflinePointAddRectSumMaintainer2D.h}
\lstinputlisting[style=cpp]{../../src/DataStruct/ClassicProblem/OfflinePointAddRectSumMaintainer2D.h}
\subsection{count\_inversion.cpp}
\lstinputlisting[style=cpp]{../../src/DataStruct/ClassicProblem/count_inversion.cpp}
\subsection{Fenwick.hpp}
\lstinputlisting[style=cpp]{../../src/DataStruct/Fenwick/Fenwick.hpp}
\subsection{BIT2D\_ex.h}
\lstinputlisting[style=cpp]{../../src/DataStruct/Fenwick/BIT2D_ex.h}
\subsection{SegTree.h}
\lstinputlisting[style=cpp]{../../src/DataStruct/Segtree/SegTree.h}
\subsection{AssignZkwTree.h}
\lstinputlisting[style=cpp]{../../src/DataStruct/Segtree/AssignZkwTree.h}
\subsection{PersistentCompressedTree.h}
\lstinputlisting[style=cpp]{../../src/DataStruct/Segtree/PersistentCompressedTree.h}
\subsection{SegBit.h}
\lstinputlisting[style=cpp]{../../src/DataStruct/TreeTree/SegBit.h}
\section{Geometry}
\subsection{Line.cpp}
\lstinputlisting[style=cpp]{../../src/Geometry/Line.cpp}
\subsection{geometry.cpp}
\lstinputlisting[style=cpp]{../../src/Geometry/geometry.cpp}
\subsection{PolygonAndConvex.cpp}
\lstinputlisting[style=cpp]{../../src/Geometry/PolygonAndConvex.cpp}
\subsection{Point.cpp}
\lstinputlisting[style=cpp]{../../src/Geometry/Point.cpp}
\subsection{Triangle.cpp}
\lstinputlisting[style=cpp]{../../src/Geometry/Triangle.cpp}
\subsection{Circle.cpp}
\lstinputlisting[style=cpp]{../../src/Geometry/Circle.cpp}
\subsection{HalfPlane.cpp}
\lstinputlisting[style=cpp]{../../src/Geometry/HalfPlane.cpp}
\section{Graph}
\subsection{EulerPath\_dg.h}
\lstinputlisting[style=cpp]{../../src/Graph/EulerPath_dg.h}
\subsection{graph\_segment.cpp}
\lstinputlisting[style=cpp]{../../src/Graph/graph_segment.cpp}
\subsection{EulerPath\_udg.h}
\lstinputlisting[style=cpp]{../../src/Graph/EulerPath_udg.h}
\subsection{Tarjan\_scc.h}
\lstinputlisting[style=cpp]{../../src/Graph/connectivity/Tarjan_scc.h}
\subsection{TwoSat.h}
\lstinputlisting[style=cpp]{../../src/Graph/connectivity/TwoSat.h}
\subsection{Tarjan\_bridge.h}
\lstinputlisting[style=cpp]{../../src/Graph/connectivity/Tarjan_bridge.h}
\subsection{Tarjan\_cut.h}
\lstinputlisting[style=cpp]{../../src/Graph/connectivity/Tarjan_cut.h}
\subsection{HopcroftKarp.h}
\lstinputlisting[style=cpp]{../../src/Graph/match/HopcroftKarp.h}
\subsection{KuhnMunkres.h}
\lstinputlisting[style=cpp]{../../src/Graph/match/KuhnMunkres.h}
\subsection{FloydWarshall.h}
\lstinputlisting[style=cpp]{../../src/Graph/shortest-path/FloydWarshall.h}
\subsection{Johnson.h}
\lstinputlisting[style=cpp]{../../src/Graph/shortest-path/Johnson.h}
\subsection{Dijkstra.h}
\lstinputlisting[style=cpp]{../../src/Graph/shortest-path/Dijkstra.h}
\subsection{find\_cycle.cpp}
\lstinputlisting[style=cpp]{../../src/Graph/shortest-path/find_cycle.cpp}
\subsection{SPFA.h}
\lstinputlisting[style=cpp]{../../src/Graph/shortest-path/SPFA.h}
\subsection{circle3count.cpp}
\lstinputlisting[style=cpp]{../../src/Graph/ClassicProblem/circle3count.cpp}
\subsection{Dinic\_mcmf.h}
\lstinputlisting[style=cpp]{../../src/Graph/flow/Dinic_mcmf.h}
\subsection{Dinic.h}
\lstinputlisting[style=cpp]{../../src/Graph/flow/Dinic.h}
\section{Math}
\subsection{power.hpp}
\lstinputlisting[style=cpp]{../../src/Math/Misc/power.hpp}
\subsection{Frac.cpp}
\lstinputlisting[style=cpp]{../../src/Math/Misc/Frac.cpp}
\subsection{LucasTable.h}
\lstinputlisting[style=cpp]{../../src/Math/Combination/LucasTable.h}
\subsection{CombinationTable.h}
\lstinputlisting[style=cpp]{../../src/Math/Combination/CombinationTable.h}
\subsection{cantor.cpp}
\lstinputlisting[style=cpp]{../../src/Math/Combination/cantor.cpp}
\subsection{CatalanNumber.h}
\lstinputlisting[style=cpp]{../../src/Math/Combination/CatalanNumber.h}
\subsection{PollardRho.h}
\lstinputlisting[style=cpp]{../../src/Math/NumberTheory/PollardRho.h}
\subsection{exEuler.cpp}
\lstinputlisting[style=cpp]{../../src/Math/NumberTheory/exEuler.cpp}
\subsection{EulerSieve.h}
\lstinputlisting[style=cpp]{../../src/Math/NumberTheory/EulerSieve.h}
\subsection{Mobius.h}
\lstinputlisting[style=cpp]{../../src/Math/NumberTheory/Mobius.h}
\subsection{SqrtDecomposition.h}
\lstinputlisting[style=cpp]{../../src/Math/NumberTheory/SqrtDecomposition.h}
\subsection{ExtendedEuclidean.h}
\lstinputlisting[style=cpp]{../../src/Math/NumberTheory/ExtendedEuclidean.h}
\subsection{PrimeCheck.h}
\lstinputlisting[style=cpp]{../../src/Math/NumberTheory/PrimeCheck.h}
\subsection{LinearModEquations.h}
\lstinputlisting[style=cpp]{../../src/Math/NumberTheory/LinearModEquations.h}
\subsection{ChineseRemainderTheorem\_ex.h}
\lstinputlisting[style=cpp]{../../src/Math/NumberTheory/ChineseRemainderTheorem_ex.h}
\subsection{Fib.cpp}
\lstinputlisting[style=cpp]{../../src/Math/ClassicProblem/Fib.cpp}
\subsection{StaticModInt32.h}
\lstinputlisting[style=cpp]{../../src/Math/Modular/StaticModInt32.h}
\subsection{DynamicModInt32.h}
\lstinputlisting[style=cpp]{../../src/Math/Modular/DynamicModInt32.h}
\subsection{Modular.h}
\lstinputlisting[style=cpp]{../../src/Math/Modular/Modular.h}
\subsection{StaticModInt64.h}
\lstinputlisting[style=cpp]{../../src/Math/Modular/StaticModInt64.h}
\subsection{DynamicModInt64.h}
\lstinputlisting[style=cpp]{../../src/Math/Modular/DynamicModInt64.h}
\subsection{HamelXorBase.h}
\lstinputlisting[style=cpp]{../../src/Math/LinearAlgebra/HamelXorBase.h}
\subsection{GaussJordanElimination.h}
\lstinputlisting[style=cpp]{../../src/Math/LinearAlgebra/GaussJordanElimination.h}
\subsection{GaussJordanBitxorElimination.h}
\lstinputlisting[style=cpp]{../../src/Math/LinearAlgebra/GaussJordanBitxorElimination.h}
\subsection{StaticMatrix.h}
\lstinputlisting[style=cpp]{../../src/Math/LinearAlgebra/StaticMatrix.h}
\subsection{LagrangeInterpolation.h}
\lstinputlisting[style=cpp]{../../src/Math/Poly/LagrangeInterpolation.h}
\subsection{NTTPolynomial.h}
\lstinputlisting[style=cpp]{../../src/Math/Poly/NTTPolynomial.h}
\subsection{FFTPolynomial.h}
\lstinputlisting[style=cpp]{../../src/Math/Poly/FFTPolynomial.h}
\section{Misc}
\subsection{longest\_ascending\_subsequence.cpp}
\lstinputlisting[style=cpp]{../../src/Misc/longest_ascending_subsequence.cpp}
\subsection{杜教BM.cpp}
\lstinputlisting[style=cpp]{../../src/Misc/杜教BM.cpp}
\subsection{interval\_merging.cpp}
\lstinputlisting[style=cpp]{../../src/Misc/interval_merging.cpp}
\subsection{Digit\_dp.cpp}
\lstinputlisting[style=cpp]{../../src/Misc/Digit_dp.cpp}
\subsection{tree\_bag.cpp}
\lstinputlisting[style=cpp]{../../src/Misc/tree_bag.cpp}
\subsection{std\_bit.h}
\lstinputlisting[style=cpp]{../../src/Misc/std_bit.h}
\subsection{Simulated\_annealing.cpp}
\lstinputlisting[style=cpp]{../../src/Misc/Simulated_annealing.cpp}
\section{String}
\subsection{FastSequenceAutomaton.h}
\lstinputlisting[style=cpp]{../../src/String/FastSequenceAutomaton.h}
\subsection{MinimalSequence.h}
\lstinputlisting[style=cpp]{../../src/String/MinimalSequence.h}
\subsection{PolyHash.cpp}
\lstinputlisting[style=cpp]{../../src/String/PolyHash.cpp}
\subsection{SequenceAutomaton.h}
\lstinputlisting[style=cpp]{../../src/String/SequenceAutomaton.h}
\subsection{trie.cpp}
\lstinputlisting[style=cpp]{../../src/String/trie.cpp}
\subsection{Manacher.h}
\lstinputlisting[style=cpp]{../../src/String/Manacher.h}
\subsection{kmp.cpp}
\lstinputlisting[style=cpp]{../../src/String/kmp.cpp}
\subsection{SequenceHash.h}
\lstinputlisting[style=cpp]{../../src/String/SequenceHash.h}
\subsection{Z\_function.cpp}
\lstinputlisting[style=cpp]{../../src/String/Z_function.cpp}
\section{Tree}
\subsection{RaySeg.h}
\lstinputlisting[style=cpp]{../../src/Tree/RaySeg.h}
\subsection{VectorTree.h}
\lstinputlisting[style=cpp]{../../src/Tree/VectorTree.h}
\subsection{Centroid.h}
\lstinputlisting[style=cpp]{../../src/Tree/Centroid.h}
\subsection{HeavyLightDecomposition.h}
\lstinputlisting[style=cpp]{../../src/Tree/HeavyLightDecomposition.h}
\subsection{RMQLCA.h}
\lstinputlisting[style=cpp]{../../src/Tree/RMQLCA.h}
\subsection{VirtualTree.h}
\lstinputlisting[style=cpp]{../../src/Tree/VirtualTree.h}
\subsection{treeHash.cpp}
\lstinputlisting[style=cpp]{../../src/Tree/treeHash.cpp}
\subsection{Prufer.h}
\lstinputlisting[style=cpp]{../../src/Tree/Prufer.h}


\end{document}
\label{LastPage}